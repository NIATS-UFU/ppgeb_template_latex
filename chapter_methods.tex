\chapter{Materials and Methods}
\label{ChapterMethods}

\section{Listed Items}

\begin{itemize}
    \item \lipsum[1][1-3]
    \item \lipsum[1][1-3]
    \item \lipsum[1][1-3]
\end{itemize}

\section{Numbered Items}

\begin{enumerate}
    \item \lipsum[1][1-3]
    \item \lipsum[1][1-3]
    \item \lipsum[1][1-3]
\end{enumerate}

\section{Nested enumeration numbering}

\begin{enumerate}
    \item \lipsum[1][1-3]
    \begin{enumerate}
        \item \lipsum[1][1-3]
        \begin{enumerate}
            \item \lipsum[1][1-3]
        \end{enumerate}
        \item \lipsum[1][1-3]
    \end{enumerate}
    \item \lipsum[1][1-3]
\end{enumerate}

\section{Equations}

The approximate entropy is described in Equation \ref{eq:ApEn}. Another example is Equation \ref{eq:MyEquation}.

\begin{equation}
\label{eq:ApEn}
ApEn(m,r,N) = \left\{\begin{array}{rc}
\phi^m(r) - \phi^{m+1}(r), & \forall \; m\le 0\\
- \phi^1(r), & m = 0
\end{array}\right.,
\end{equation}

\begin{equation}
\label{eq:MyEquation}
x=\frac{-b\pm\sqrt{b^2-4ac}}{2a}.
\end{equation}

\section{Pictures}

You can create geometric representations in \LaTeX.

\begin{center}
\setlength{\unitlength}{1mm}
\begin{picture}(50,25)
\put(0,0){\scriptsize$C$}
\put(2,2){\circle*{0.7}}
\put(2,2){\vector(1,0){50}} 
\put(52,2){\circle*{0.7}}
\put(52,0){\scriptsize$B$}
\put(2,2){\vector(1,1){20}} 
\put(22,22){\circle*{0.7}}
\put(22,22.5){\scriptsize$A$}
\put(2,2){\vector(2,1){28.5}} 
\put(34,16){\scriptsize$X$}
\put(30.5,16.25){\circle*{0.7}}
\put(22,22){\line(3,-2){30}} 
\end{picture}
\end{center}

\section{Code}

You can insert scripts into your text.

\subsection{R}

The code below illustrates a code snippet in the R language \cite{RCT2021}.

\begin{minted}{R}
# Creating a Graph
attach(mtcars)
plot(wt, mpg)
abline(lm(mpg~wt))
title("Regression of MPG on Weight")
\end{minted}

% \subsection{Python}

% \begin{lstlisting}[language=Python, caption=Python example]
% import numpy as np
    
% def incmatrix(genl1,genl2):
%     m = len(genl1)
%     n = len(genl2)
%     M = None #to become the incidence matrix
%     VT = np.zeros((n*m,1), int)  #dummy variable
    
%     #compute the bitwise xor matrix
%     M1 = bitxormatrix(genl1)
%     M2 = np.triu(bitxormatrix(genl2),1) 

%     for i in range(m-1):
%         for j in range(i+1, m):
%             [r,c] = np.where(M2 == M1[i,j])
%             for k in range(len(r)):
%                 VT[(i)*n + r[k]] = 1;
%                 VT[(i)*n + c[k]] = 1;
%                 VT[(j)*n + r[k]] = 1;
%                 VT[(j)*n + c[k]] = 1;
                
%                 if M is None:
%                     M = np.copy(VT)
%                 else:
%                     M = np.concatenate((M, VT), 1)
                
%                 VT = np.zeros((n*m,1), int)
    
%     return M
% \end{lstlisting}

\section{Algorithms}

Algorithm \ref{alg:cap} is an example.

\begin{algorithm}
\caption{An algorithm with caption}\label{alg:cap}
\begin{algorithmic}
\Require $n \geq 0$
\Ensure $y = x^n$
\State $y \gets 1$
\State $X \gets x$
\State $N \gets n$
\While{$N \neq 0$}
\If{$N$ is even}
    \State $X \gets X \times X$
    \State $N \gets \frac{N}{2}$  \Comment{This is a comment}
\ElsIf{$N$ is odd}
    \State $y \gets y \times X$
    \State $N \gets N - 1$
\EndIf
\EndWhile
\end{algorithmic}
\end{algorithm}


\section{References}

This template uses the ABNT standards to present the references. You can use the number style or the author-date style.