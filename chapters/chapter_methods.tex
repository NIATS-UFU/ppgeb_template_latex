\chapter{Materials and Methods}
\label{ChapterMethods}

\section{Listed Items}

\begin{itemize}
    \item \lipsum[1][1-3]
    \item \lipsum[1][1-3]
    \item \lipsum[1][1-3]
\end{itemize}

\section{Numbered Items}

\begin{enumerate}
    \item \lipsum[1][1-3]
    \item \lipsum[1][1-3]
    \item \lipsum[1][1-3]
\end{enumerate}

\section{Nested enumeration numbering}

\begin{enumerate}
    \item \lipsum[1][1-3]
    \begin{enumerate}
        \item \lipsum[1][1-3]
        \begin{enumerate}
            \item \lipsum[1][1-3]
        \end{enumerate}
        \item \lipsum[1][1-3]
    \end{enumerate}
    \item \lipsum[1][1-3]
\end{enumerate}

\section{Equations}

\gls{ApEn} is described in Equation \ref{eq:ApEn}. Another example of an equation in \LaTeX is Equation \ref{eq:MyEquation}.

\begin{equation}
\label{eq:ApEn}
ApEn(m,r,N) = \left\{\begin{array}{rc}
\phi^m(r) - \phi^{m+1}(r), & \forall \; m\le 0\\
- \phi^1(r), & m = 0
\end{array}\right.,
\end{equation}

\begin{equation}
\label{eq:MyEquation}
x=\frac{-b\pm\sqrt{b^2-4ac}}{2a}.
\end{equation}
 
 \LaTeX provides a wide range of symbols for mathematical notation, including Greek letters (Table \ref{tab:GreekLettters}), operators (Table \ref{tab:CommonSymbols}), and special characters. These symbols are typically used within \textbf{math mode}, which can be activated using dollar signs (\verb|$ ... $|) for inline expressions or double dollar signs (\verb|$$ ... $$|) or environments like \verb|\[ ... \]| for displayed equations.
 
\begin{table}[h!]
\centering
\renewcommand{\arraystretch}{1.2}
\rowcolors{2}{gray!10}{white}
\begin{tabularx}{0.85\textwidth}{>{\centering\arraybackslash}m{0.18\textwidth} X >{\centering\arraybackslash}m{0.18\textwidth} X}
\toprule
\textbf{Symbol} & \textbf{Command} & \textbf{Symbol} & \textbf{Command} \\
\midrule
$\alpha$  & \verb|\alpha|  & $\beta$  & \verb|\beta| \\
$\gamma$  & \verb|\gamma|  & $\delta$ & \verb|\delta| \\
$\epsilon$ & \verb|\epsilon| & $\zeta$ & \verb|\zeta| \\
$\eta$ & \verb|\eta| & $\theta$ & \verb|\theta| \\
$\iota$ & \verb|\iota| & $\kappa$ & \verb|\kappa| \\
$\lambda$ & \verb|\lambda| & $\mu$ & \verb|\mu| \\
$\nu$ & \verb|\nu| & $\xi$ & \verb|\xi| \\
$o$ (omicron) & \verb|$o$| & $\pi$ & \verb|\pi| \\
$\rho$ & \verb|\rho| & $\sigma$ & \verb|\sigma| \\
$\tau$ & \verb|\tau| & $\upsilon$ & \verb|\upsilon| \\
$\phi$ & \verb|\phi| & $\chi$ & \verb|\chi| \\
$\psi$ & \verb|\psi| & $\omega$ & \verb|\omega| \\
\midrule
$\Gamma$ & \verb|\Gamma| & $\Delta$ & \verb|\Delta| \\
$\Theta$ & \verb|\Theta| & $\Lambda$ & \verb|\Lambda| \\
$\Xi$ & \verb|\Xi| & $\Pi$ & \verb|\Pi| \\
$\Sigma$ & \verb|\Sigma| & $\Upsilon$ & \verb|\Upsilon| \\
$\Phi$ & \verb|\Phi| & $\Psi$ & \verb|\Psi| \\
$\Omega$ & \verb|\Omega| & & \\
\bottomrule
\end{tabularx}
\caption{Greek letters and their corresponding  \LaTeX commands.}\label{tab:GreekLettters}
\end{table}


% --- Common Mathematical Symbols Table ---
\begin{table}[h!]
\centering
\renewcommand{\arraystretch}{1.2}
\rowcolors{2}{gray!10}{white}
\begin{tabularx}{0.85\textwidth}{
    >{\centering\arraybackslash}m{0.18\textwidth} 
    X 
    >{\centering\arraybackslash}m{0.18\textwidth} 
    X
}
\toprule
\textbf{Symbol} & \textbf{Command} & \textbf{Symbol} & \textbf{Command} \\
\midrule
$\times$ & \verb|\times| & $\div$ & \verb|\div| \\
$\pm$ & \verb|\pm| & $\mp$ & \verb|\mp| \\
$\cdot$ & \verb|\cdot| & $\ast$ & \verb|\ast| \\
$\leq$ & \verb|\leq| & $\geq$ & \verb|\geq| \\
$\neq$ & \verb|\neq| & $\approx$ & \verb|\approx| \\
$\equiv$ & \verb|\equiv| & $\propto$ & \verb|\propto| \\
$\infty$ & \verb|\infty| & $\partial$ & \verb|\partial| \\
$\nabla$ & \verb|\nabla| & $\forall$ & \verb|\forall| \\
$\exists$ & \verb|\exists| & $\nexists$ & \verb|\nexists| \\
$\in$ & \verb|\in| & $\notin$ & \verb|\notin| \\
$\subseteq$ & \verb|\subseteq| & $\supseteq$ & \verb|\supseteq| \\
$\subset$ & \verb|\subset| & $\supset$ & \verb|\supset| \\
$\cup$ & \verb|\cup| & $\cap$ & \verb|\cap| \\
$\Rightarrow$ & \verb|\Rightarrow| & $\Leftrightarrow$ & \verb|\Leftrightarrow| \\
$\uparrow$ & \verb|\uparrow| & $\downarrow$ & \verb|\downarrow| \\
\bottomrule
\end{tabularx}
\caption{Common mathematical symbols and their corresponding \LaTeX{} commands.}
\label{tab:CommonSymbols}
\end{table}


\section{Pictures}

You can create geometric representations in \LaTeX.

\begin{center}
\setlength{\unitlength}{1mm}
\begin{picture}(50,25)
\put(0,0){\scriptsize$C$}
\put(2,2){\circle*{0.7}}
\put(2,2){\vector(1,0){50}} 
\put(52,2){\circle*{0.7}}
\put(52,0){\scriptsize$B$}
\put(2,2){\vector(1,1){20}} 
\put(22,22){\circle*{0.7}}
\put(22,22.5){\scriptsize$A$}
\put(2,2){\vector(2,1){28.5}} 
\put(34,16){\scriptsize$X$}
\put(30.5,16.25){\circle*{0.7}}
\put(22,22){\line(3,-2){30}} 
\end{picture}
\end{center}

\section{Code}

You can insert scripts into your text.

\subsection{R}

The code below illustrates a code snippet in the R language \cite{RCT2021}.

\begin{lstlisting}[language=R]
# loading R package
library(ggplot2)

# data visualization
ggplot(mpg, aes(displ, hwy, colour = class)) +
  geom_point()
\end{lstlisting}


\subsection{Python}

\begin{lstlisting}[language=Python]
  # libraries:
  import numpy as np
  import pandas as pd
  import matplotlib.pyplot as plt
  # creating the variables:
  x = np.linspace(0,100)
  y = x**2
  # plotting
  plt.plot(x, y, '-b')
\end{lstlisting}

\subsection{SQL}

\begin{lstlisting}[language=SQL]
  SELECT 
     id,
     name,
     age,
     sex,
     disease
  FROM participants
  ORDER BY 1000;
\end{lstlisting}

\subsection{C}

\begin{lstlisting}[language=C]
#include <stdio.h>

int main() {
    printf("Hello, world!\n");
    return 0;
}
\end{lstlisting}

\subsection{Matlab}

\begin{lstlisting}[language=Matlab]
% sampling frequency (Hz)
fs = 1000;

% time vector (2 seconds)
t = 0:1/fs:2;

% signal frequency (Hz)
f_signal = 10;

% sinusoidal signal
signal = sin(2*pi*f_signal*t);

% additive white Gaussian noise
noise = 0.4 * randn(size(t));

% resulting noisy signal
x = signal + noise;
\end{lstlisting}


\section{Algorithms}

Algorithm \ref{alg:cap} is an example.

\begin{algorithm}
\caption{An algorithm with caption}\label{alg:cap}
\begin{algorithmic}
\Require $n \geq 0$
\Ensure $y = x^n$
\State $y \gets 1$
\State $X \gets x$
\State $N \gets n$
\While{$N \neq 0$}
\If{$N$ is even}
    \State $X \gets X \times X$
    \State $N \gets \frac{N}{2}$  \Comment{This is a comment}
\ElsIf{$N$ is odd}
    \State $y \gets y \times X$
    \State $N \gets N - 1$
\EndIf
\EndWhile
\end{algorithmic}
\end{algorithm}


\section{References}

This template uses the ABNT standards to present the references. You can use the number style or the author-date style.
